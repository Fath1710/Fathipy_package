
\documentclass{article}
\usepackage{listings}
\usepackage{geometry}
\geometry{a4paper, margin=1in}
\title{First-Package: Estructura y Documentación}
\begin{document}
\maketitle
\section{Estructura del Paquete}
La estructura ideal para el paquete \texttt{First-Package} es la siguiente:

\begin{verbatim}
First-Package/
├── First-Package/
│   ├── __init__.py
│   ├── module1.py
│   ├── module2.py
├── tests/
│   ├── __init__.py
│   ├── test_module1.py
│   └── test_module2.py
├── README.md
├── setup.py
├── LICENSE
├── .gitignore
└── requirements.txt
\end{verbatim}

\section{Descripción de Archivos}
\begin{itemize}
    \item \texttt{First-Package/\_\_init\_\_.py}: Inicializa el paquete \texttt{First-Package}.
    \item \texttt{First-Package/module1.py} y \texttt{First-Package/module2.py}: Contienen la lógica y funciones del paquete.
    \item \texttt{tests/}: Directorio que contiene los archivos de prueba para cada módulo.
    \item \texttt{README.md}: Documentación básica del paquete.
    \item \texttt{setup.py}: Script de configuración para instalar el paquete.
    \item \texttt{LICENSE}: Documento de licencia del paquete.
    \item \texttt{.gitignore}: Archivos y carpetas que Git debería ignorar.
    \item \texttt{requirements.txt}: Lista de dependencias del paquete.
\end{itemize}

\section{Instalación y Ejecución}
Para instalar este paquete de manera local:

\begin{verbatim}
pip install -e .
\end{verbatim}

Para ejecutar funciones:

\begin{verbatim}
from First-Package import module1
module1.example_function()
\end{verbatim}

\section{Uso de Git y GitHub en Visual Studio Code}
\begin{enumerate}
    \item Inicializar Git en el terminal de VS Code: \texttt{git init}.
    \item Conectar con GitHub: Crear un repositorio y agregarlo como remoto.
    \item Añadir y confirmar cambios: \texttt{git add .}, \texttt{git commit -m "mensaje"}.
    \item Subir el código: \texttt{git push -u origin main}.
\end{enumerate}

\section{Uso del Paquete por un Usuario Externo}
Un usuario puede instalar el paquete ejecutando:

\begin{verbatim}
pip install git+https://github.com/usuario/First-Package.git
\end{verbatim}

Luego puede usarlo como:

\begin{verbatim}
from First-Package import module1
module1.example_function()
\end{verbatim}

\end{document}
