
\documentclass{article}
\usepackage{listings}
\usepackage{geometry}
\geometry{a4paper, margin=1in}
\title{Fathipy: Estructura y Documentación}
\begin{document}
\maketitle
\section{Estructura del Paquete}
La estructura ideal para el paquete \texttt{Fathipy} es la siguiente:

\begin{verbatim}
Fathipy/
├── Fathipy/
│   ├── __init__.py
│   ├── three_vector.py
│   ├── rotation.py
│   ├── SMparticle.py
├── tests/
│   ├── __init__.py
│   ├── test_threevector.py
│   ├── test_rotation.py
│   ├── test_SMparticle.py
├── README.md
├── setup.py
├── LICENSE
├── .gitignore
└── requirements.txt
\end{verbatim}

\section{Descripción de Archivos}
\begin{itemize}
    \item \texttt{Fathipy/\_\_init\_\_.py}: Inicializa el paquete \texttt{Fathipy}.
    \item \texttt{Fathipy/three_vector.py}, \texttt{Fathipy/rotation.py} y \texttt{Fathipy/SMparticle.py}: Contienen la lógica y funciones principales del paquete, como la manipulación de vectores, rotaciones y partículas del modelo estándar.
    \item \texttt{tests/}: Directorio que contiene los archivos de prueba para cada módulo.
    \item \texttt{README.md}: Documentación básica del paquete, incluyendo instrucciones de instalación y ejemplos de uso.
    \item \texttt{setup.py}: Script de configuración para instalar el paquete.
    \item \texttt{LICENSE}: Documento de licencia del paquete.
    \item \texttt{.gitignore}: Archivos y carpetas que Git debería ignorar.
    \item \texttt{requirements.txt}: Lista de dependencias del paquete.
\end{itemize}

\section{Instalación y Ejecución}
Para instalar este paquete de manera local:

\begin{verbatim}
pip install -e .
\end{verbatim}

\section{Uso}

\subsection{ThreeVector}
La clase \texttt{ThreeVector} permite trabajar con vectores tridimensionales y realizar operaciones matemáticas sobre ellos.

\textbf{Ejemplo de uso:}
\begin{verbatim}
from Fathipy.three_vector import ThreeVector

# Crear un vector
v1 = ThreeVector(1, 2, 3)

# Crear otro vector
v2 = ThreeVector(4, 5, 6)

# Sumar vectores
v3 = v1 + v2
print(v3)  # ThreeVector(x=5, y=7, z=9)

# Producto cruzado
v4 = v1.cross(v2)
print(v4)  # ThreeVector(x=-3, y=6, z=-3)

# Producto punto
dot_product = v1.dot(v2)
print(dot_product)  # 32
\end{verbatim}

\subsection{Rotation}
La clase \texttt{Rotation} permite aplicar rotaciones 3D sobre vectores utilizando la fórmula de Rodrigues.

\textbf{Ejemplo de uso:}
\begin{verbatim}
from Fathipy.rotation import Rotation
from Fathipy.three_vector import ThreeVector
import numpy as np

# Crear un eje de rotación
axis = ThreeVector(0, 0, 1)  # Eje Z

# Crear una rotación de 90 grados (π/2 radianes) alrededor del eje Z
rotation = Rotation(np.pi / 2, axis)

# Crear un vector
v = ThreeVector(1, 0, 0)

# Aplicar la rotación
v_rotated = rotation.apply(v)
print(v_rotated)  # ThreeVector(x=0.0, y=1.0, z=0.0)
\end{verbatim}

\subsection{StandardModelParticle}
La clase \texttt{StandardModelParticle} representa una partícula del modelo estándar con propiedades como su nombre, símbolo, masa, carga, espín, tipo de partícula y si es una antipartícula.

\textbf{Ejemplo de uso:}
\begin{verbatim}
from Fathipy.SMparticle import StandardModelParticle

# Crear una partícula
electron = StandardModelParticle(
    name="Electron",
    symbol="e-",
    mass=0.511,
    charge=-1,
    spin=0.5,
    type_of_particle="fermion"
)

# Obtener la descripción de la partícula
print(electron.info())
# Output: Electron (e-) - Masa: 0.511 MeV/c^2, Carga: -1, Espín: 0.5, Tipo: fermion, Antipartícula: False
\end{verbatim}

\section{Pruebas}
El paquete incluye pruebas unitarias para verificar que las funcionalidades estén implementadas correctamente. Para ejecutar las pruebas, usa el siguiente comando:

\begin{verbatim}
python -m unittest discover -s tests
\end{verbatim}

Este comando buscará y ejecutará todas las pruebas del paquete. Las pruebas cubren la correcta inicialización de las clases, las operaciones matemáticas y la manipulación de partículas.

\section{Licencia}
Este proyecto está bajo la Licencia MIT. Para más detalles, consulta el archivo \texttt{LICENSE} en el repositorio.

\section{Uso del Paquete por un Usuario Externo}
Un usuario puede instalar el paquete ejecutando:

\begin{verbatim}
pip install git+https://github.com/Fath1710/Fathipy_package.git
\end{verbatim}

\end{document}
